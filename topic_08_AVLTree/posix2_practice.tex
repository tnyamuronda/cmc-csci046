\documentclass[10pt]{article}

\usepackage[margin=1in]{geometry}
\usepackage{amsmath}
\usepackage{amssymb}
\usepackage{amsthm}
\usepackage{mathtools}
\usepackage[shortlabels]{enumitem}
\usepackage[normalem]{ulem}
\usepackage{courier}

\usepackage{hyperref}
\hypersetup{
  colorlinks   = true, %Colours links instead of ugly boxes
  urlcolor     = black, %Colour for external hyperlinks
  linkcolor    = blue, %Colour of internal links
  citecolor    = blue  %Colour of citations
}

\usepackage[T1]{fontenc}
\usepackage{upquote}
\usepackage{listings}
\lstset{
    language=HTML
    ,basicstyle=\linespread{1}\ttfamily
    ,keywordstyle=
    ,language=sh
    ,showstringspaces=false
    ,numbers=left
    ,breaklines=true
    }

%%%%%%%%%%%%%%%%%%%%%%%%%%%%%%%%%%%%%%%%%%%%%%%%%%%%%%%%%%%%%%%%%%%%%%%%%%%%%%%%

\theoremstyle{definition}
\newtheorem{problem}{Problem}
\newtheorem{note}{Note}
\newcommand{\E}{\mathbb E}
\newcommand{\R}{\mathbb R}
\DeclareMathOperator{\Var}{Var}
\DeclareMathOperator*{\argmin}{arg\,min}
\DeclareMathOperator*{\argmax}{arg\,max}

\newcommand{\trans}[1]{{#1}^{T}}
\newcommand{\loss}{\ell}
\newcommand{\w}{\mathbf w}
\newcommand{\mle}[1]{\hat{#1}_{\textit{mle}}}
\newcommand{\map}[1]{\hat{#1}_{\textit{map}}}
\newcommand{\normal}{\mathcal{N}}
\newcommand{\x}{\mathbf x}
\newcommand{\y}{\mathbf y}
\newcommand{\ltwo}[1]{\lVert {#1} \rVert}

%%%%%%%%%%%%%%%%%%%%%%%%%%%%%%%%%%%%%%%%%%%%%%%%%%%%%%%%%%%%%%%%%%%%%%%%%%%%%%%%

\begin{document}
\begin{center}
    {
\Large
    Quiz: POSIX Shell II (Practice Problems)
}

    \vspace{0.1in}
\end{center}

\section{If statements}

\filbreak
\begin{problem}
    Write the output of the final command in the following terminal session.
    If the command has no output, then leave the problem blank.
\end{problem}
\begin{lstlisting}
$ cd; rm -rf quiz; mkdir quiz; cd quiz
$ foo='hola'
$ cat > quiz.sh <<'EOF'
foo='hello'
if [ $foo = "hello" ]; then
    touch if
fi
EOF
$ sh quiz.sh
$ ls
\end{lstlisting}


\filbreak
\begin{problem}
    Write the output of the final command in the following terminal session.
    If the command has no output, then leave the problem blank.
\end{problem}
\begin{lstlisting}
$ cd; rm -rf quiz; mkdir quiz; cd quiz
$ foo='hola'
$ cat > quiz.sh <<'EOF'
foo='hello'
if [ $foo != "hello" ]; then
    touch if
fi
EOF
$ sh quiz.sh
$ ls
\end{lstlisting}

\filbreak
\begin{problem}
    Write the output of the final command in the following terminal session.
    If the command has no output, then leave the problem blank.
\end{problem}
\begin{lstlisting}
$ cd; rm -rf quiz; mkdir quiz; cd quiz
$ foo='hola'
$ cat > quiz.sh <<EOF
foo='hello'
if [ $foo = "hello" ]; then
    touch if
fi
EOF
$ sh quiz.sh
$ ls
\end{lstlisting}


\filbreak
\begin{problem}
    Write the output of the final command in the following terminal session.
    If the command has no output, then leave the problem blank.
\end{problem}
\begin{lstlisting}
$ cd; rm -rf quiz; mkdir quiz; cd quiz
$ foo='hola'
$ cat > quiz.sh <<'EOF'
foo='hello world'
if [ $foo = "hello" ]; then
    touch if
fi
EOF
$ sh quiz.sh
$ ls
\end{lstlisting}


\filbreak
\subsection{The \lstinline{else} and \lstinline{elif} keywords}

\begin{problem}
    Write the output of the final command in the following terminal session.
    If the command has no output, then leave the problem blank.
\end{problem}
\begin{lstlisting}
$ cd; rm -rf quiz; mkdir quiz; cd quiz
$ foo='hola'
$ cat > quiz.sh <<'EOF'
foo='hello world'
if [ $foo = "hello" ]; then
    touch if
else
    touch else
fi
EOF
$ sh quiz.sh
$ ls
\end{lstlisting}


\filbreak
\begin{problem}
    Write the output of the final command in the following terminal session.
    If the command has no output, then leave the problem blank.
\end{problem}
\begin{lstlisting}
$ cd; rm -rf quiz; mkdir quiz; cd quiz
$ foo='hola'
$ cat > quiz.sh <<EOF
foo='hello'
if [ "$foo" = "hello" ]; then
    touch if
elif [ "$foo" = "hola" ]; then
    touch elif
else
    touch else
fi
EOF
$ sh quiz.sh
$ ls
\end{lstlisting}


\filbreak
\begin{problem}
    Write the output of the final command in the following terminal session.
    If the command has no output, then leave the problem blank.
\end{problem}
\begin{lstlisting}
$ cd; rm -rf quiz; mkdir quiz; cd quiz
$ foo='hola'
$ cat > quiz.sh <<EOF
foo='hello'
if [ "$foo" = "hello" ]; then
    touch if
elif [ "$foo" = "hola" ]; then
    touch elif
else
    touch else
fi
EOF
$ sh quiz.sh
$ ls
\end{lstlisting}


\filbreak
\subsection{\lstinline{&&} and \lstinline{||}}


\filbreak
\begin{problem}
    Write the output of the final command in the following terminal session.
    If the command has no output, then leave the problem blank.
\end{problem}
\begin{lstlisting}
$ cd; rm -rf quiz; mkdir quiz; cd quiz
$ foo='hola'
$ cat > quiz.sh <<EOF
foo='hello'
if [ "$foo" = "hello" ] || [ "$foo" = "hola" ]; then
    touch if
else
    touch else
fi
EOF
$ sh quiz.sh
$ ls
\end{lstlisting}


\filbreak
\begin{problem}
    Write the output of the final command in the following terminal session.
    If the command has no output, then leave the problem blank.
\end{problem}
\begin{lstlisting}
$ cd; rm -rf quiz; mkdir quiz; cd quiz
$ foo='hola'
$ cat > quiz.sh <<'EOF'
foo='hello'
bar='salve'
if [ "$foo" = "hello" ] && [ "$bar" = "salve" ]; then
    touch if
else
    touch else
fi
EOF
$ sh quiz.sh
$ ls
\end{lstlisting}

\filbreak
\begin{problem}
    Write the output of the final command in the following terminal session.
    If the command has no output, then leave the problem blank.
\end{problem}
\begin{lstlisting}
$ cd; rm -rf quiz; mkdir quiz; cd quiz
$ foo='hola'
$ cat > quiz.sh <<'EOF'
foo='hello'
bar='salve'
if true && [ "$bar" = "salve" ]; then
    touch if
else
    touch else
fi
EOF
$ sh quiz.sh
$ ls
\end{lstlisting}

\filbreak
\begin{problem}
    Write the output of the final command in the following terminal session.
    If the command has no output, then leave the problem blank.
\end{problem}
\begin{lstlisting}
$ cd; rm -rf quiz; mkdir quiz; cd quiz
$ foo='hola'
$ cat > quiz.sh <<'EOF'
foo='hello'
bar='salve'
if false || ([ "$bar" = "salve" ] && true); then
    touch if
else
    touch else
fi
EOF
$ sh quiz.sh
$ ls
\end{lstlisting}

\filbreak
\subsection{The \lstinline{!} operator}

\begin{problem}
    Write the output of the final command in the following terminal session.
    If the command has no output, then leave the problem blank.
\end{problem}
\begin{lstlisting}
$ cd; rm -rf quiz; mkdir quiz; cd quiz
$ foo='hola'
$ cat > quiz.sh <<'EOF'
foo='hello'
if ! [ $foo = "hello" ]; then
    touch if
fi
EOF
$ sh quiz.sh
$ ls
\end{lstlisting}

\filbreak
\begin{problem}
    Write the output of the final command in the following terminal session.
    If the command has no output, then leave the problem blank.
\end{problem}
\begin{lstlisting}
$ cd; rm -rf quiz; mkdir quiz; cd quiz
$ foo='hola'
$ cat > quiz.sh <<'EOF'
foo='hello'
if ! [ $foo != "hello" ]; then
    touch if
fi
EOF
$ sh quiz.sh
$ ls
\end{lstlisting}

\filbreak
\begin{problem}
    Write the output of the final command in the following terminal session.
    If the command has no output, then leave the problem blank.
\end{problem}
\begin{lstlisting}
$ cd; rm -rf quiz; mkdir quiz; cd quiz
$ foo='hola'
$ cat > quiz.sh <<'EOF'
foo='hello'
bar='salve'
if ! true || [ "$bar" != "salve" ]; then
    touch if
else
    touch else
fi
EOF
$ sh quiz.sh
$ ls
\end{lstlisting}

\section{Inline conditions}

\filbreak
\begin{problem}
    Write the output of the final command in the following terminal session.
    If the command has no output, then leave the problem blank.
\end{problem}
\begin{lstlisting}
$ cd; rm -rf quiz; mkdir quiz; cd quiz
$ foo='hola'
$ false || echo $foo > false
$ true || echo $foo > true
$ ls
\end{lstlisting}


\filbreak
\begin{problem}
    Write the output of the final command in the following terminal session.
    If the command has no output, then leave the problem blank.
\end{problem}
\begin{lstlisting}
$ cd; rm -rf quiz; mkdir quiz; cd quiz
$ foo='hola'
$ false && echo $foo > false
$ true && echo $foo > true
$ ls
\end{lstlisting}

\filbreak
\begin{problem}
    Write the output of the final command in the following terminal session.
    If the command has no output, then leave the problem blank.
\end{problem}
\begin{lstlisting}
$ cd; rm -rf quiz; mkdir quiz; cd quiz
$ foo='hola'
$ (false && echo $foo) > false
$ (true && echo $foo) > true
$ ls
\end{lstlisting}

\filbreak
\begin{problem}
    Write the output of the final command in the following terminal session.
    If the command has no output, then leave the problem blank.
\end{problem}
\begin{lstlisting}
$ cd; rm -rf quiz; mkdir quiz; cd quiz
$ foo='hola'
$ [ "$foo" = 'hello' ] && echo $foo > false
$ ls
\end{lstlisting}


\filbreak
\begin{problem}
    Write the output of the final command in the following terminal session.
    If the command has no output, then leave the problem blank.
\end{problem}
\begin{lstlisting}
$ cd; rm -rf quiz; mkdir quiz; cd quiz
$ foo='hola'
$ [ "$foo" = 'hello' ] || echo $foo > false
$ ls
\end{lstlisting}

\filbreak
\begin{problem}
    Write the output of the final command in the following terminal session.
    If the command has no output, then leave the problem blank.
\end{problem}
\begin{lstlisting}
$ cd; rm -rf quiz; mkdir quiz; cd quiz
$ foo='hola'
$ ! [ "$foo" = 'hello' ] || echo $foo > false
$ ls
\end{lstlisting}

\section{Exit codes}

\filbreak
\begin{problem}
    Write the output of the final command in the following terminal session.
    If the command has no output, then leave the problem blank.
\end{problem}
\begin{lstlisting}
$ cd; rm -rf quiz; mkdir quiz; cd quiz
$ cat > logs <<EOF
INFO: blah
INFO: blah
ERROR: blah blah blah
INFO: blah
EOF
$ cat logs | grep 'ERROR' || echo 'hello world' > foo
$ ls
\end{lstlisting}

\filbreak
\begin{problem}
    Write the output of the final command in the following terminal session.
    If the command has no output, then leave the problem blank.
\end{problem}
\begin{lstlisting}
$ cd; rm -rf quiz; mkdir quiz; cd quiz
$ cat > logs <<EOF
INFO: blah
INFO: blah
WARNING: blah blah blah
INFO: blah
EOF
$ cat logs | grep 'ERROR' || echo 'hello world' > foo
$ ls
\end{lstlisting}

\filbreak
\begin{problem}
    Write the output of the final command in the following terminal session.
    If the command has no output, then leave the problem blank.
\end{problem}
\begin{lstlisting}
$ cd; rm -rf quiz; mkdir quiz; cd quiz
$ cat > logs <<EOF
INFO: blah
INFO: blah
WARNING: blah blah blah
INFO: blah
EOF
$ cat logs | grep 'ERROR' && echo 'hello world' > foo
$ ls
\end{lstlisting}

\filbreak
\begin{problem}
    Write the output of the final command in the following terminal session.
    If the command has no output, then leave the problem blank.
\end{problem}
\begin{lstlisting}
$ cd; rm -rf quiz; mkdir quiz; cd quiz
$ cat > logs <<EOF
INFO: blah
INFO: blah
ERROR: blah blah blah
INFO: blah
EOF
$ cat logs | grep 'ERROR' && echo 'hello world' > foo
$ ls
\end{lstlisting}


\filbreak
\begin{problem}
    Write the output of the final command in the following terminal session.
    If the command has no output, then leave the problem blank.
\end{problem}
\begin{lstlisting}
$ cd; rm -rf quiz; mkdir quiz; cd quiz
$ cat > logs <<EOF
INFO: blah
INFO: blah
ERROR: blah blah blah
INFO: blah
EOF
$ cat logs | grep 'ERROR' > /dev/null && echo 'hello world' > foo
$ ls
\end{lstlisting}
\subsection{Commands inside if}

%\filbreak
%\begin{problem}
    %Write the output of the final command in the following terminal session.
    %If the command has no output, then leave the problem blank.
%\end{problem}
%\begin{lstlisting}
%$ cd; rm -rf quiz; mkdir quiz; cd quiz
%$ cat > logs <<EOF
%INFO: blah
%INFO: blah
%ERROR: blah blah blah
%INFO: blah
%EOF
%$ cat > quiz.sh <<'EOF'
%if cat logs | grep ERROR; then
    %touch if
%fi
%EOF
%$ sh quiz.sh
%$ ls
%\end{lstlisting}

\filbreak
\begin{problem}
    Write the output of the final command in the following terminal session.
    If the command has no output, then leave the problem blank.
\end{problem}
\begin{lstlisting}
$ cd; rm -rf quiz; mkdir quiz; cd quiz
$ cat > logs <<EOF
INFO: blah
INFO: blah
ERROR: blah blah blah
INFO: blah
EOF
$ cat > quiz.sh <<'EOF'
if cat logs | grep ERROR > /dev/null; then
    touch if
fi
EOF
$ sh quiz.sh
$ ls
\end{lstlisting}


\filbreak
\begin{problem}
    Write the output of the final command in the following terminal session.
    If the command has no output, then leave the problem blank.
\end{problem}
\begin{lstlisting}
$ cd; rm -rf quiz; mkdir quiz; cd quiz
$ cat > logs <<EOF
INFO: blah
INFO: blah
WARNING: blah blah blah
INFO: blah
EOF
$ cat > quiz.sh <<'EOF'
if cat logs | grep ERROR > /dev/null; then
    touch error
elif cat logs | grep WARNING > /dev/null; then
    touch warning
elif cat logs | grep INFO > /dev/null; then
    touch info
fi
EOF
$ sh quiz.sh
$ ls
\end{lstlisting}

\filbreak
\end{document}
