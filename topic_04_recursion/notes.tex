\documentclass[10pt]{article}

\usepackage[margin=1in]{geometry}
\usepackage{amsmath}
\usepackage{amssymb}
\usepackage{amsthm}
\usepackage{mathtools}
\usepackage[shortlabels]{enumitem}
\usepackage[normalem]{ulem}
\usepackage{courier}

\usepackage{hyperref}
\hypersetup{
  colorlinks   = true, %Colours links instead of ugly boxes
  urlcolor     = black, %Colour for external hyperlinks
  linkcolor    = blue, %Colour of internal links
  citecolor    = blue  %Colour of citations
}

\usepackage[T1]{fontenc}
\usepackage{upquote}
\usepackage{listings}
\lstset{
    language=HTML
    ,basicstyle=\linespread{1}\ttfamily
    ,keywordstyle=
    ,language=sh
    ,showstringspaces=false
    ,numbers=left
    ,breaklines=true
    }


%%%%%%%%%%%%%%%%%%%%%%%%%%%%%%%%%%%%%%%%%%%%%%%%%%%%%%%%%%%%%%%%%%%%%%%%%%%%%%%%

\theoremstyle{definition}
\newtheorem{problem}{Problem}
\newtheorem{note}{Note}
\newcommand{\E}{\mathbb E}
\newcommand{\R}{\mathbb R}
\DeclareMathOperator{\Var}{Var}
\DeclareMathOperator*{\argmin}{arg\,min}
\DeclareMathOperator*{\argmax}{arg\,max}

\newcommand{\trans}[1]{{#1}^{T}}
\newcommand{\loss}{\ell}
\newcommand{\w}{\mathbf w}
\newcommand{\mle}[1]{\hat{#1}_{\textit{mle}}}
\newcommand{\map}[1]{\hat{#1}_{\textit{map}}}
\newcommand{\normal}{\mathcal{N}}
\newcommand{\x}{\mathbf x}
\newcommand{\y}{\mathbf y}
\newcommand{\ltwo}[1]{\lVert {#1} \rVert}

%%%%%%%%%%%%%%%%%%%%%%%%%%%%%%%%%%%%%%%%%%%%%%%%%%%%%%%%%%%%%%%%%%%%%%%%%%%%%%%%

\begin{document}
\begin{center}
    {
\Large
    Notes.py
}

    \vspace{0.1in}
\end{center}

\section{Sequential Search}

\filbreak
\begin{lstlisting}
def sequential_search_itr(xs, y):
    '''
    Check whether y is contained in the list xs

    >>> sequential_search_itr([1, 3, 5, 4, 2, 0], 2)
    True
    >>> sequential_search_itr([1, 3, 5, 4, 2, 0], 6)
    False
    '''
    for x in xs:
        if x == y:
            return True
    return False
\end{lstlisting}

\filbreak
\begin{lstlisting}
def sequential_search_itr2(xs, y):
    '''
    Check whether y is contained in the list xs

    >>> sequential_search_itr2([1, 3, 5, 4, 2, 0], 2)
    True
    >>> sequential_search_itr2([1, 3, 5, 4, 2, 0], 6)
    False
    '''
    for i in range(len(xs)):
        if xs[i] == y:
            return True
    return False
\end{lstlisting}

\filbreak
\begin{lstlisting}
def sequential_search_rec(xs, y):
    '''
    Check whether y is contained in the list xs

    >>> sequential_search_rec([1, 3, 5, 4, 2, 0], 2)
    True
    >>> sequential_search_rec([1, 3, 5, 4, 2, 0], 6)
    False
    '''
    if len(xs) == 0:
        return False
    if xs[0] == y:
        return True
    return sequential_search_rec(xs[1:], y)
\end{lstlisting}

\filbreak
\begin{lstlisting}
def sequential_search_rec2(xs, y):
    '''
    Check whether y is contained in the list xs

    >>> sequential_search_rec2([1, 3, 5, 4, 2, 0], 2)
    True
    >>> sequential_search_rec2([1, 3, 5, 4, 2, 0], 6)
    False
    '''
    def go(xs):
        if len(xs) == 0:
            return False
        if xs[0] == y:
            return True
        return go(xs[1:])
    return go(xs)
\end{lstlisting}

\filbreak
\begin{lstlisting}
def sequential_search_rec3(xs, y):
    '''
    Check whether y is contained in the list xs

    >>> sequential_search_rec3([1, 3, 5, 4, 2, 0], 2)
    True
    >>> sequential_search_rec3([1, 3, 5, 4, 2, 0], 6)
    False
    '''
    def go(i):
        if i == len(xs):
            return False
        if xs[i] == y:
            return True
        return go(i+1)
    return go(0)
\end{lstlisting}

\filbreak
\section{Binary Search}

\begin{lstlisting}
def binary_search_itr(xs, y):
    '''
    Assume that xs is a list of numbers sorted from LOWEST to HIGHEST.
    Return True if y is in the list xs.

    >>> binary_search_itr([1, 3, 5, 7, 9, 11], 9)
    True
    >>> binary_search_itr([1, 3, 5, 7, 9, 11], 8)
    False
    >>> binary_search_itr(list(range(-1001, 1001, 2)), 9)
    True
    >>> binary_search_itr(list(range(-1000, 1000, 2)), 9)
    False
    '''
    if len(xs) == 0:
        return False
    left = 0
    right = len(xs) - 1

    while left != right:
        mid = (left + right) // 2
        if xs[mid] > y:
            right = mid
        if xs[mid] < y:
            left = mid + 1
        if xs[mid] == y:
            return True
            left = mid + 1

    if xs[left] == y:
        return True
    else:
        return False

    return go(0, len(xs) - 1)
\end{lstlisting}

\filbreak
\begin{lstlisting}
def binary_search_rec(xs, y):
    '''
    Assume that xs is a list of numbers sorted from LOWEST to HIGHEST.
    Return True if y is in the list xs.

    >>> binary_search_rec([1, 3, 5, 7, 9, 11], 9)
    True
    >>> binary_search_rec([1, 3, 5, 7, 9, 11], 8)
    False
    >>> binary_search_rec(list(range(-1001, 1001, 2)), 9)
    True
    >>> binary_search_rec(list(range(-1000, 1000, 2)), 9)
    False
    '''
    if len(xs) == 0:
        return False

    def go(left, right):
        if left == right:
            if xs[left] == y:
                return True
            else:
                return False
        mid = (left + right) // 2
        if xs[mid] > y:
            right = mid
        if xs[mid] < y:
            left = mid + 1
        if xs[mid] == y:
            return True
            left = mid + 1
        return go(left, right)

    return go(0, len(xs) - 1)
\end{lstlisting}

\filbreak
\begin{lstlisting}
def binary_search_rec2(xs, y):
    '''
    Assume that xs is a list of numbers sorted from LOWEST to HIGHEST.
    Return True if y is in the list xs.

    >>> binary_search_rec2([1, 3, 5, 7, 9, 11], 9)
    True
    >>> binary_search_rec2([1, 3, 5, 7, 9, 11], 8)
    False
    >>> binary_search_rec2(list(range(-1001, 1001, 2)), 9)
    True
    >>> binary_search_rec2(list(range(-1000, 1000, 2)), 9)
    False
    '''
    if len(xs) == 0:
        return False

    mid = len(xs) // 2
    if xs[mid] > y:
        return binary_search_rec2(xs[:mid], y)
    if xs[mid] < y:
        return binary_search_rec2(xs[mid+1:], y)
    if xs[mid] == y:
        return True
\end{lstlisting}

\end{document}
