\documentclass[10pt]{article}

\usepackage[margin=1in]{geometry}
\usepackage{amsmath}
\usepackage{amssymb}
\usepackage{amsthm}
\usepackage{mathtools}
\usepackage[shortlabels]{enumitem}
\usepackage[normalem]{ulem}
\usepackage{courier}

\usepackage{hyperref}
\hypersetup{
  colorlinks   = true, %Colours links instead of ugly boxes
  urlcolor     = black, %Colour for external hyperlinks
  linkcolor    = blue, %Colour of internal links
  citecolor    = blue  %Colour of citations
}

\usepackage[T1]{fontenc}
\usepackage{upquote}
\usepackage{listings}
\lstset{
    language=HTML
    ,basicstyle=\linespread{1}\ttfamily
    ,keywordstyle=
    ,language=sh
    ,showstringspaces=false
    ,numbers=left
    ,breaklines=true
    }


%%%%%%%%%%%%%%%%%%%%%%%%%%%%%%%%%%%%%%%%%%%%%%%%%%%%%%%%%%%%%%%%%%%%%%%%%%%%%%%%

\theoremstyle{definition}
\newtheorem{problem}{Problem}
\newtheorem{note}{Note}
\newcommand{\E}{\mathbb E}
\newcommand{\R}{\mathbb R}
\DeclareMathOperator{\Var}{Var}
\DeclareMathOperator*{\argmin}{arg\,min}
\DeclareMathOperator*{\argmax}{arg\,max}

\newcommand{\trans}[1]{{#1}^{T}}
\newcommand{\loss}{\ell}
\newcommand{\w}{\mathbf w}
\newcommand{\mle}[1]{\hat{#1}_{\textit{mle}}}
\newcommand{\map}[1]{\hat{#1}_{\textit{map}}}
\newcommand{\normal}{\mathcal{N}}
\newcommand{\x}{\mathbf x}
\newcommand{\y}{\mathbf y}
\newcommand{\ltwo}[1]{\lVert {#1} \rVert}

%%%%%%%%%%%%%%%%%%%%%%%%%%%%%%%%%%%%%%%%%%%%%%%%%%%%%%%%%%%%%%%%%%%%%%%%%%%%%%%%

\begin{document}
\begin{center}
    {
\Large
    Sorting Notes
}
\end{center}

\section{Merge Sort}

\begin{lstlisting}
def merge_sorted(xs):
    '''
    Returns the input list xs in sorted order.
    '''
    if len(xs) <= 1:
        return xs
    else:
        mid = len(xs) // 2
        left = xs[:mid]
        right = xs[mid:]
        return _merged(merge_sorted(left), merge_sorted(right))


def _merged(xs, ys):
    '''
    Assuming xs and ys are sorted lists,
    returns a sorted list containing the elements of both xs and ys.
    Runs in linear time.
    '''
\end{lstlisting}

\newpage
\section{Quick Sort}

\begin{lstlisting}
def quick_sorted(xs):
    '''
    Returns the input list xs in sorted order.
    '''
    if len(xs) <= 1:
        return xs
    mid = len(xs) // 2
    pivot = xs[mid]
    xs_smaller = [x for x in xs if x < pivot]
    xs_bigger  = [x for x in xs if x > pivot]
    xs_equal   = [x for x in xs if x == pivot]
    return quick_sorted(xs_smaller) + xs_equal + quick_sorted(xs_greater)
\end{lstlisting}


\end{document}
